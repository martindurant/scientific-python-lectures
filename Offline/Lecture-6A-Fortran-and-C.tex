
% Default to the notebook output style

    


% Inherit from the specified cell style.




    
\documentclass{article}

    
    
    \usepackage{graphicx} % Used to insert images
    \usepackage{adjustbox} % Used to constrain images to a maximum size 
    \usepackage{color} % Allow colors to be defined
    \usepackage{enumerate} % Needed for markdown enumerations to work
    \usepackage{geometry} % Used to adjust the document margins
    \usepackage{amsmath} % Equations
    \usepackage{amssymb} % Equations
    \usepackage{eurosym} % defines \euro
    \usepackage[mathletters]{ucs} % Extended unicode (utf-8) support
    \usepackage[utf8x]{inputenc} % Allow utf-8 characters in the tex document
    \usepackage{fancyvrb} % verbatim replacement that allows latex
    \usepackage{grffile} % extends the file name processing of package graphics 
                         % to support a larger range 
    % The hyperref package gives us a pdf with properly built
    % internal navigation ('pdf bookmarks' for the table of contents,
    % internal cross-reference links, web links for URLs, etc.)
    \usepackage{hyperref}
    \usepackage{longtable} % longtable support required by pandoc >1.10
    \usepackage{booktabs}  % table support for pandoc > 1.12.2
    

    
    
    \definecolor{orange}{cmyk}{0,0.4,0.8,0.2}
    \definecolor{darkorange}{rgb}{.71,0.21,0.01}
    \definecolor{darkgreen}{rgb}{.12,.54,.11}
    \definecolor{myteal}{rgb}{.26, .44, .56}
    \definecolor{gray}{gray}{0.45}
    \definecolor{lightgray}{gray}{.95}
    \definecolor{mediumgray}{gray}{.8}
    \definecolor{inputbackground}{rgb}{.95, .95, .85}
    \definecolor{outputbackground}{rgb}{.95, .95, .95}
    \definecolor{traceback}{rgb}{1, .95, .95}
    % ansi colors
    \definecolor{red}{rgb}{.6,0,0}
    \definecolor{green}{rgb}{0,.65,0}
    \definecolor{brown}{rgb}{0.6,0.6,0}
    \definecolor{blue}{rgb}{0,.145,.698}
    \definecolor{purple}{rgb}{.698,.145,.698}
    \definecolor{cyan}{rgb}{0,.698,.698}
    \definecolor{lightgray}{gray}{0.5}
    
    % bright ansi colors
    \definecolor{darkgray}{gray}{0.25}
    \definecolor{lightred}{rgb}{1.0,0.39,0.28}
    \definecolor{lightgreen}{rgb}{0.48,0.99,0.0}
    \definecolor{lightblue}{rgb}{0.53,0.81,0.92}
    \definecolor{lightpurple}{rgb}{0.87,0.63,0.87}
    \definecolor{lightcyan}{rgb}{0.5,1.0,0.83}
    
    % commands and environments needed by pandoc snippets
    % extracted from the output of `pandoc -s`
    \providecommand{\tightlist}{%
      \setlength{\itemsep}{0pt}\setlength{\parskip}{0pt}}
    \DefineVerbatimEnvironment{Highlighting}{Verbatim}{commandchars=\\\{\}}
    % Add ',fontsize=\small' for more characters per line
    \newenvironment{Shaded}{}{}
    \newcommand{\KeywordTok}[1]{\textcolor[rgb]{0.00,0.44,0.13}{\textbf{{#1}}}}
    \newcommand{\DataTypeTok}[1]{\textcolor[rgb]{0.56,0.13,0.00}{{#1}}}
    \newcommand{\DecValTok}[1]{\textcolor[rgb]{0.25,0.63,0.44}{{#1}}}
    \newcommand{\BaseNTok}[1]{\textcolor[rgb]{0.25,0.63,0.44}{{#1}}}
    \newcommand{\FloatTok}[1]{\textcolor[rgb]{0.25,0.63,0.44}{{#1}}}
    \newcommand{\CharTok}[1]{\textcolor[rgb]{0.25,0.44,0.63}{{#1}}}
    \newcommand{\StringTok}[1]{\textcolor[rgb]{0.25,0.44,0.63}{{#1}}}
    \newcommand{\CommentTok}[1]{\textcolor[rgb]{0.38,0.63,0.69}{\textit{{#1}}}}
    \newcommand{\OtherTok}[1]{\textcolor[rgb]{0.00,0.44,0.13}{{#1}}}
    \newcommand{\AlertTok}[1]{\textcolor[rgb]{1.00,0.00,0.00}{\textbf{{#1}}}}
    \newcommand{\FunctionTok}[1]{\textcolor[rgb]{0.02,0.16,0.49}{{#1}}}
    \newcommand{\RegionMarkerTok}[1]{{#1}}
    \newcommand{\ErrorTok}[1]{\textcolor[rgb]{1.00,0.00,0.00}{\textbf{{#1}}}}
    \newcommand{\NormalTok}[1]{{#1}}
    
    % Define a nice break command that doesn't care if a line doesn't already
    % exist.
    \def\br{\hspace*{\fill} \\* }
    % Math Jax compatability definitions
    \def\gt{>}
    \def\lt{<}
    % Document parameters
    \title{Lecture-6A-Fortran-and-C}
    
    
    

    % Pygments definitions
    
\makeatletter
\def\PY@reset{\let\PY@it=\relax \let\PY@bf=\relax%
    \let\PY@ul=\relax \let\PY@tc=\relax%
    \let\PY@bc=\relax \let\PY@ff=\relax}
\def\PY@tok#1{\csname PY@tok@#1\endcsname}
\def\PY@toks#1+{\ifx\relax#1\empty\else%
    \PY@tok{#1}\expandafter\PY@toks\fi}
\def\PY@do#1{\PY@bc{\PY@tc{\PY@ul{%
    \PY@it{\PY@bf{\PY@ff{#1}}}}}}}
\def\PY#1#2{\PY@reset\PY@toks#1+\relax+\PY@do{#2}}

\expandafter\def\csname PY@tok@ni\endcsname{\let\PY@bf=\textbf\def\PY@tc##1{\textcolor[rgb]{0.60,0.60,0.60}{##1}}}
\expandafter\def\csname PY@tok@si\endcsname{\let\PY@bf=\textbf\def\PY@tc##1{\textcolor[rgb]{0.73,0.40,0.53}{##1}}}
\expandafter\def\csname PY@tok@c\endcsname{\let\PY@it=\textit\def\PY@tc##1{\textcolor[rgb]{0.25,0.50,0.50}{##1}}}
\expandafter\def\csname PY@tok@gs\endcsname{\let\PY@bf=\textbf}
\expandafter\def\csname PY@tok@ow\endcsname{\let\PY@bf=\textbf\def\PY@tc##1{\textcolor[rgb]{0.67,0.13,1.00}{##1}}}
\expandafter\def\csname PY@tok@gd\endcsname{\def\PY@tc##1{\textcolor[rgb]{0.63,0.00,0.00}{##1}}}
\expandafter\def\csname PY@tok@vc\endcsname{\def\PY@tc##1{\textcolor[rgb]{0.10,0.09,0.49}{##1}}}
\expandafter\def\csname PY@tok@nv\endcsname{\def\PY@tc##1{\textcolor[rgb]{0.10,0.09,0.49}{##1}}}
\expandafter\def\csname PY@tok@nt\endcsname{\let\PY@bf=\textbf\def\PY@tc##1{\textcolor[rgb]{0.00,0.50,0.00}{##1}}}
\expandafter\def\csname PY@tok@kd\endcsname{\let\PY@bf=\textbf\def\PY@tc##1{\textcolor[rgb]{0.00,0.50,0.00}{##1}}}
\expandafter\def\csname PY@tok@cp\endcsname{\def\PY@tc##1{\textcolor[rgb]{0.74,0.48,0.00}{##1}}}
\expandafter\def\csname PY@tok@cm\endcsname{\let\PY@it=\textit\def\PY@tc##1{\textcolor[rgb]{0.25,0.50,0.50}{##1}}}
\expandafter\def\csname PY@tok@nn\endcsname{\let\PY@bf=\textbf\def\PY@tc##1{\textcolor[rgb]{0.00,0.00,1.00}{##1}}}
\expandafter\def\csname PY@tok@nf\endcsname{\def\PY@tc##1{\textcolor[rgb]{0.00,0.00,1.00}{##1}}}
\expandafter\def\csname PY@tok@gi\endcsname{\def\PY@tc##1{\textcolor[rgb]{0.00,0.63,0.00}{##1}}}
\expandafter\def\csname PY@tok@s2\endcsname{\def\PY@tc##1{\textcolor[rgb]{0.73,0.13,0.13}{##1}}}
\expandafter\def\csname PY@tok@gt\endcsname{\def\PY@tc##1{\textcolor[rgb]{0.00,0.27,0.87}{##1}}}
\expandafter\def\csname PY@tok@m\endcsname{\def\PY@tc##1{\textcolor[rgb]{0.40,0.40,0.40}{##1}}}
\expandafter\def\csname PY@tok@sb\endcsname{\def\PY@tc##1{\textcolor[rgb]{0.73,0.13,0.13}{##1}}}
\expandafter\def\csname PY@tok@sh\endcsname{\def\PY@tc##1{\textcolor[rgb]{0.73,0.13,0.13}{##1}}}
\expandafter\def\csname PY@tok@sr\endcsname{\def\PY@tc##1{\textcolor[rgb]{0.73,0.40,0.53}{##1}}}
\expandafter\def\csname PY@tok@il\endcsname{\def\PY@tc##1{\textcolor[rgb]{0.40,0.40,0.40}{##1}}}
\expandafter\def\csname PY@tok@kr\endcsname{\let\PY@bf=\textbf\def\PY@tc##1{\textcolor[rgb]{0.00,0.50,0.00}{##1}}}
\expandafter\def\csname PY@tok@cs\endcsname{\let\PY@it=\textit\def\PY@tc##1{\textcolor[rgb]{0.25,0.50,0.50}{##1}}}
\expandafter\def\csname PY@tok@nl\endcsname{\def\PY@tc##1{\textcolor[rgb]{0.63,0.63,0.00}{##1}}}
\expandafter\def\csname PY@tok@s1\endcsname{\def\PY@tc##1{\textcolor[rgb]{0.73,0.13,0.13}{##1}}}
\expandafter\def\csname PY@tok@gp\endcsname{\let\PY@bf=\textbf\def\PY@tc##1{\textcolor[rgb]{0.00,0.00,0.50}{##1}}}
\expandafter\def\csname PY@tok@vg\endcsname{\def\PY@tc##1{\textcolor[rgb]{0.10,0.09,0.49}{##1}}}
\expandafter\def\csname PY@tok@mi\endcsname{\def\PY@tc##1{\textcolor[rgb]{0.40,0.40,0.40}{##1}}}
\expandafter\def\csname PY@tok@vi\endcsname{\def\PY@tc##1{\textcolor[rgb]{0.10,0.09,0.49}{##1}}}
\expandafter\def\csname PY@tok@gr\endcsname{\def\PY@tc##1{\textcolor[rgb]{1.00,0.00,0.00}{##1}}}
\expandafter\def\csname PY@tok@mh\endcsname{\def\PY@tc##1{\textcolor[rgb]{0.40,0.40,0.40}{##1}}}
\expandafter\def\csname PY@tok@err\endcsname{\def\PY@bc##1{\setlength{\fboxsep}{0pt}\fcolorbox[rgb]{1.00,0.00,0.00}{1,1,1}{\strut ##1}}}
\expandafter\def\csname PY@tok@na\endcsname{\def\PY@tc##1{\textcolor[rgb]{0.49,0.56,0.16}{##1}}}
\expandafter\def\csname PY@tok@bp\endcsname{\def\PY@tc##1{\textcolor[rgb]{0.00,0.50,0.00}{##1}}}
\expandafter\def\csname PY@tok@nd\endcsname{\def\PY@tc##1{\textcolor[rgb]{0.67,0.13,1.00}{##1}}}
\expandafter\def\csname PY@tok@nc\endcsname{\let\PY@bf=\textbf\def\PY@tc##1{\textcolor[rgb]{0.00,0.00,1.00}{##1}}}
\expandafter\def\csname PY@tok@sc\endcsname{\def\PY@tc##1{\textcolor[rgb]{0.73,0.13,0.13}{##1}}}
\expandafter\def\csname PY@tok@no\endcsname{\def\PY@tc##1{\textcolor[rgb]{0.53,0.00,0.00}{##1}}}
\expandafter\def\csname PY@tok@mo\endcsname{\def\PY@tc##1{\textcolor[rgb]{0.40,0.40,0.40}{##1}}}
\expandafter\def\csname PY@tok@nb\endcsname{\def\PY@tc##1{\textcolor[rgb]{0.00,0.50,0.00}{##1}}}
\expandafter\def\csname PY@tok@ge\endcsname{\let\PY@it=\textit}
\expandafter\def\csname PY@tok@o\endcsname{\def\PY@tc##1{\textcolor[rgb]{0.40,0.40,0.40}{##1}}}
\expandafter\def\csname PY@tok@gu\endcsname{\let\PY@bf=\textbf\def\PY@tc##1{\textcolor[rgb]{0.50,0.00,0.50}{##1}}}
\expandafter\def\csname PY@tok@s\endcsname{\def\PY@tc##1{\textcolor[rgb]{0.73,0.13,0.13}{##1}}}
\expandafter\def\csname PY@tok@gh\endcsname{\let\PY@bf=\textbf\def\PY@tc##1{\textcolor[rgb]{0.00,0.00,0.50}{##1}}}
\expandafter\def\csname PY@tok@ss\endcsname{\def\PY@tc##1{\textcolor[rgb]{0.10,0.09,0.49}{##1}}}
\expandafter\def\csname PY@tok@kt\endcsname{\def\PY@tc##1{\textcolor[rgb]{0.69,0.00,0.25}{##1}}}
\expandafter\def\csname PY@tok@w\endcsname{\def\PY@tc##1{\textcolor[rgb]{0.73,0.73,0.73}{##1}}}
\expandafter\def\csname PY@tok@sd\endcsname{\let\PY@it=\textit\def\PY@tc##1{\textcolor[rgb]{0.73,0.13,0.13}{##1}}}
\expandafter\def\csname PY@tok@sx\endcsname{\def\PY@tc##1{\textcolor[rgb]{0.00,0.50,0.00}{##1}}}
\expandafter\def\csname PY@tok@mb\endcsname{\def\PY@tc##1{\textcolor[rgb]{0.40,0.40,0.40}{##1}}}
\expandafter\def\csname PY@tok@k\endcsname{\let\PY@bf=\textbf\def\PY@tc##1{\textcolor[rgb]{0.00,0.50,0.00}{##1}}}
\expandafter\def\csname PY@tok@ne\endcsname{\let\PY@bf=\textbf\def\PY@tc##1{\textcolor[rgb]{0.82,0.25,0.23}{##1}}}
\expandafter\def\csname PY@tok@kn\endcsname{\let\PY@bf=\textbf\def\PY@tc##1{\textcolor[rgb]{0.00,0.50,0.00}{##1}}}
\expandafter\def\csname PY@tok@c1\endcsname{\let\PY@it=\textit\def\PY@tc##1{\textcolor[rgb]{0.25,0.50,0.50}{##1}}}
\expandafter\def\csname PY@tok@go\endcsname{\def\PY@tc##1{\textcolor[rgb]{0.53,0.53,0.53}{##1}}}
\expandafter\def\csname PY@tok@se\endcsname{\let\PY@bf=\textbf\def\PY@tc##1{\textcolor[rgb]{0.73,0.40,0.13}{##1}}}
\expandafter\def\csname PY@tok@kp\endcsname{\def\PY@tc##1{\textcolor[rgb]{0.00,0.50,0.00}{##1}}}
\expandafter\def\csname PY@tok@kc\endcsname{\let\PY@bf=\textbf\def\PY@tc##1{\textcolor[rgb]{0.00,0.50,0.00}{##1}}}
\expandafter\def\csname PY@tok@mf\endcsname{\def\PY@tc##1{\textcolor[rgb]{0.40,0.40,0.40}{##1}}}

\def\PYZbs{\char`\\}
\def\PYZus{\char`\_}
\def\PYZob{\char`\{}
\def\PYZcb{\char`\}}
\def\PYZca{\char`\^}
\def\PYZam{\char`\&}
\def\PYZlt{\char`\<}
\def\PYZgt{\char`\>}
\def\PYZsh{\char`\#}
\def\PYZpc{\char`\%}
\def\PYZdl{\char`\$}
\def\PYZhy{\char`\-}
\def\PYZsq{\char`\'}
\def\PYZdq{\char`\"}
\def\PYZti{\char`\~}
% for compatibility with earlier versions
\def\PYZat{@}
\def\PYZlb{[}
\def\PYZrb{]}
\makeatother


    % Exact colors from NB
    \definecolor{incolor}{rgb}{0.0, 0.0, 0.5}
    \definecolor{outcolor}{rgb}{0.545, 0.0, 0.0}



    
    % Prevent overflowing lines due to hard-to-break entities
    \sloppy 
    % Setup hyperref package
    \hypersetup{
      breaklinks=true,  % so long urls are correctly broken across lines
      colorlinks=true,
      urlcolor=blue,
      linkcolor=darkorange,
      citecolor=darkgreen,
      }
    % Slightly bigger margins than the latex defaults
    
    \geometry{verbose,tmargin=1in,bmargin=1in,lmargin=1in,rmargin=1in}
    
    

    \begin{document}
    
    
    \maketitle
    
    

    
    \begin{figure}[htbp]
\centering
\includegraphics{./images/Continuum_Logo_0702.png}
\caption{Continuum Logo}
\end{figure}

    \section{Using Fortran and C code with
Python}\label{using-fortran-and-c-code-with-python}

    This curriculum builds on material by J. Robert Johansson from his
``Introduction to scientific computing with Python,'' generously made
available under a
\href{http://creativecommons.org/licenses/by/3.0/}{Creative Commons
Attribution 3.0 Unported License} at
https://github.com/jrjohansson/scientific-python-lectures. The Continuum
Analytics enhancements use the
\href{https://creativecommons.org/licenses/by-nc/4.0/}{Creative Commons
Attribution-NonCommercial 4.0 International License}.

\begin{center}\rule{3in}{0.4pt}\end{center}

    \begin{Verbatim}[commandchars=\\\{\}]
{\color{incolor}In [{\color{incolor}147}]:} \PY{o}{\PYZpc{}}\PY{k}{pylab} inline
          \PY{k+kn}{from} \PY{n+nn}{IPython}\PY{n+nn}{.}\PY{n+nn}{display} \PY{k}{import} \PY{n}{Image}
\end{Verbatim}

    \begin{Verbatim}[commandchars=\\\{\}]
Populating the interactive namespace from numpy and matplotlib
    \end{Verbatim}

    The advantage of Python is that it is flexible and easy to program. The
time it takes to setup a new calulation is therefore short. But for
certain types of calculations Python (and any other interpreted
language) can be very slow. It is particularly iterations over large
arrays that is difficult to do efficiently.

Such calculations may be implemented in a compiled language such as C or
Fortran. In Python it is relatively easy to call out to libraries with
compiled C or Fortran code. In this lecture we will look at how to do
that.

But before we go ahead and work on optimizing anything, it is always
worthwhile to ask\ldots{}.

    \begin{Verbatim}[commandchars=\\\{\}]
{\color{incolor}In [{\color{incolor}148}]:} \PY{n}{Image}\PY{p}{(}\PY{n}{filename}\PY{o}{=}\PY{l+s}{\PYZsq{}}\PY{l+s}{images/optimizing\PYZhy{}what.png}\PY{l+s}{\PYZsq{}}\PY{p}{)}
\end{Verbatim}
\texttt{\color{outcolor}Out[{\color{outcolor}148}]:}
    
    \begin{center}
    \adjustimage{max size={0.9\linewidth}{0.9\paperheight}}{Lecture-6A-Fortran-and-C_files/Lecture-6A-Fortran-and-C_5_0.png}
    \end{center}
    { \hspace*{\fill} \\}
    

    \subsection{Fortran}\label{fortran}

    \subsubsection{F2PY}\label{f2py}

    F2PY is a program that (almost) automatically wraps fortran code for use
in Python: By using the \texttt{f2py} program we can compile fortran
code into a module that we can import in a Python program.

F2PY is a part of NumPy, but you will also need to have a fortran
compiler to run the examples below.

    \subsubsection{Example 0: scalar input, no
output}\label{example-0-scalar-input-no-output}

    \begin{Verbatim}[commandchars=\\\{\}]
{\color{incolor}In [{\color{incolor}149}]:} \PY{o}{\PYZpc{}\PYZpc{}}\PY{k}{file} hellofortran.f
          C File  hellofortran.f
                  subroutine hellofortran (n)
                  integer n
                 
                  do 100 i=0, n
                      print *, \PYZdq{}Fortran says hello\PYZdq{}
          100     continue
                  end
\end{Verbatim}

    \begin{Verbatim}[commandchars=\\\{\}]
Overwriting hellofortran.f
    \end{Verbatim}

    Generate a python module using \texttt{f2py}:

    \begin{Verbatim}[commandchars=\\\{\}]
{\color{incolor}In [{\color{incolor}150}]:} \PY{o}{!}f2py3 \PYZhy{}c \PYZhy{}m hellofortran hellofortran.f
\end{Verbatim}

    \begin{Verbatim}[commandchars=\\\{\}]
running build
running config\_cc
unifing config\_cc, config, build\_clib, build\_ext, build commands --compiler options
running config\_fc
unifing config\_fc, config, build\_clib, build\_ext, build commands --fcompiler options
running build\_src
build\_src
building extension "hellofortran" sources
f2py options: []
f2py:> /tmp/tmp\_6mh2wh9/src.linux-x86\_64-3.4/hellofortranmodule.c
creating /tmp/tmp\_6mh2wh9/src.linux-x86\_64-3.4
Reading fortran codes{\ldots}
	Reading file 'hellofortran.f' (format:fix,strict)
Post-processing{\ldots}
	Block: hellofortran
			Block: hellofortran
Post-processing (stage 2){\ldots}
Building modules{\ldots}
	Building module "hellofortran"{\ldots}
		Constructing wrapper function "hellofortran"{\ldots}
		  hellofortran(n)
	Wrote C/API module "hellofortran" to file "/tmp/tmp\_6mh2wh9/src.linux-x86\_64-3.4/hellofortranmodule.c"
  adding '/tmp/tmp\_6mh2wh9/src.linux-x86\_64-3.4/fortranobject.c' to sources.
  adding '/tmp/tmp\_6mh2wh9/src.linux-x86\_64-3.4' to include\_dirs.
copying /home/dhavide/anaconda3/lib/python3.4/site-packages/numpy/f2py/src/fortranobject.c -> /tmp/tmp\_6mh2wh9/src.linux-x86\_64-3.4
copying /home/dhavide/anaconda3/lib/python3.4/site-packages/numpy/f2py/src/fortranobject.h -> /tmp/tmp\_6mh2wh9/src.linux-x86\_64-3.4
build\_src: building npy-pkg config files
running build\_ext
customize UnixCCompiler
customize UnixCCompiler using build\_ext
customize Gnu95FCompiler
Found executable /usr/bin/gfortran
customize Gnu95FCompiler
customize Gnu95FCompiler using build\_ext
building 'hellofortran' extension
compiling C sources
C compiler: gcc -pthread -DNDEBUG -g -fwrapv -O3 -Wall -Wstrict-prototypes -fPIC

creating /tmp/tmp\_6mh2wh9/tmp
creating /tmp/tmp\_6mh2wh9/tmp/tmp\_6mh2wh9
creating /tmp/tmp\_6mh2wh9/tmp/tmp\_6mh2wh9/src.linux-x86\_64-3.4
compile options: '-I/tmp/tmp\_6mh2wh9/src.linux-x86\_64-3.4 -I/home/dhavide/anaconda3/lib/python3.4/site-packages/numpy/core/include -I/home/dhavide/anaconda3/include/python3.4m -c'
gcc: /tmp/tmp\_6mh2wh9/src.linux-x86\_64-3.4/hellofortranmodule.c
In file included from /home/dhavide/anaconda3/lib/python3.4/site-packages/numpy/core/include/numpy/ndarraytypes.h:1804:0,
                 from /home/dhavide/anaconda3/lib/python3.4/site-packages/numpy/core/include/numpy/ndarrayobject.h:17,
                 from /home/dhavide/anaconda3/lib/python3.4/site-packages/numpy/core/include/numpy/arrayobject.h:4,
                 from /tmp/tmp\_6mh2wh9/src.linux-x86\_64-3.4/fortranobject.h:13,
                 from /tmp/tmp\_6mh2wh9/src.linux-x86\_64-3.4/hellofortranmodule.c:17:
/home/dhavide/anaconda3/lib/python3.4/site-packages/numpy/core/include/numpy/npy\_1\_7\_deprecated\_api.h:15:2: warning: \#warning "Using deprecated NumPy API, disable it by " "\#defining NPY\_NO\_DEPRECATED\_API NPY\_1\_7\_API\_VERSION" [-Wcpp]
 \#warning "Using deprecated NumPy API, disable it by " \textbackslash{}
  \^{}
gcc: /tmp/tmp\_6mh2wh9/src.linux-x86\_64-3.4/fortranobject.c
In file included from /home/dhavide/anaconda3/lib/python3.4/site-packages/numpy/core/include/numpy/ndarraytypes.h:1804:0,
                 from /home/dhavide/anaconda3/lib/python3.4/site-packages/numpy/core/include/numpy/ndarrayobject.h:17,
                 from /home/dhavide/anaconda3/lib/python3.4/site-packages/numpy/core/include/numpy/arrayobject.h:4,
                 from /tmp/tmp\_6mh2wh9/src.linux-x86\_64-3.4/fortranobject.h:13,
                 from /tmp/tmp\_6mh2wh9/src.linux-x86\_64-3.4/fortranobject.c:2:
/home/dhavide/anaconda3/lib/python3.4/site-packages/numpy/core/include/numpy/npy\_1\_7\_deprecated\_api.h:15:2: warning: \#warning "Using deprecated NumPy API, disable it by " "\#defining NPY\_NO\_DEPRECATED\_API NPY\_1\_7\_API\_VERSION" [-Wcpp]
 \#warning "Using deprecated NumPy API, disable it by " \textbackslash{}
  \^{}
compiling Fortran sources
Fortran f77 compiler: /usr/bin/gfortran -Wall -g -ffixed-form -fno-second-underscore -fPIC -O3 -funroll-loops
Fortran f90 compiler: /usr/bin/gfortran -Wall -g -fno-second-underscore -fPIC -O3 -funroll-loops
Fortran fix compiler: /usr/bin/gfortran -Wall -g -ffixed-form -fno-second-underscore -Wall -g -fno-second-underscore -fPIC -O3 -funroll-loops
compile options: '-I/tmp/tmp\_6mh2wh9/src.linux-x86\_64-3.4 -I/home/dhavide/anaconda3/lib/python3.4/site-packages/numpy/core/include -I/home/dhavide/anaconda3/include/python3.4m -c'
gfortran:f77: hellofortran.f
/usr/bin/gfortran -Wall -g -Wall -g -shared /tmp/tmp\_6mh2wh9/tmp/tmp\_6mh2wh9/src.linux-x86\_64-3.4/hellofortranmodule.o /tmp/tmp\_6mh2wh9/tmp/tmp\_6mh2wh9/src.linux-x86\_64-3.4/fortranobject.o /tmp/tmp\_6mh2wh9/hellofortran.o -L/home/dhavide/anaconda3/lib -lpython3.4m -lgfortran -o ./hellofortran.cpython-34m.so
Removing build directory /tmp/tmp\_6mh2wh9
    \end{Verbatim}

    Example of a python script that use the module:

    \begin{Verbatim}[commandchars=\\\{\}]
{\color{incolor}In [{\color{incolor}151}]:} \PY{o}{\PYZpc{}\PYZpc{}}\PY{k}{file} hello.py
          import hellofortran
          
          hellofortran.hellofortran(5)
\end{Verbatim}

    \begin{Verbatim}[commandchars=\\\{\}]
Overwriting hello.py
    \end{Verbatim}

    \begin{Verbatim}[commandchars=\\\{\}]
{\color{incolor}In [{\color{incolor}152}]:} \PY{c}{\PYZsh{} run the script}
          \PY{o}{!}python hello.py
\end{Verbatim}

    \begin{Verbatim}[commandchars=\\\{\}]
Fortran says hello
 Fortran says hello
 Fortran says hello
 Fortran says hello
 Fortran says hello
 Fortran says hello
    \end{Verbatim}

    \subsubsection{Example 1: vector input and scalar
output}\label{example-1-vector-input-and-scalar-output}

    \begin{Verbatim}[commandchars=\\\{\}]
{\color{incolor}In [{\color{incolor}153}]:} \PY{o}{\PYZpc{}\PYZpc{}}\PY{k}{file} dprod.f
          
                 subroutine dprod(x, y, n)
              
                 double precision x(n), y
                 y = 1.0
              
                 do 100 i=1, n
                     y = y * x(i)
          100    continue
                 end
\end{Verbatim}

    \begin{Verbatim}[commandchars=\\\{\}]
Overwriting dprod.f
    \end{Verbatim}

    \begin{Verbatim}[commandchars=\\\{\}]
{\color{incolor}In [{\color{incolor}154}]:} \PY{o}{!}rm \PYZhy{}f dprod.pyf
          \PY{o}{!}f2py3 \PYZhy{}m dprod \PYZhy{}h dprod.pyf dprod.f
\end{Verbatim}

    \begin{Verbatim}[commandchars=\\\{\}]
Reading fortran codes{\ldots}
	Reading file 'dprod.f' (format:fix,strict)
Post-processing{\ldots}
	Block: dprod
\{\}
In: :dprod:dprod.f:dprod
vars2fortran: No typespec for argument "n".
			Block: dprod
Post-processing (stage 2){\ldots}
Saving signatures to file "./dprod.pyf"
    \end{Verbatim}

    The \texttt{f2py} program generated a module declaration file called
\texttt{dsum.pyf}. Let's look what's in it:

    \begin{Verbatim}[commandchars=\\\{\}]
{\color{incolor}In [{\color{incolor}155}]:} \PY{o}{!}cat dprod.pyf
\end{Verbatim}

    \begin{Verbatim}[commandchars=\\\{\}]
!    -*- f90 -*-
! Note: the context of this file is case sensitive.

python module dprod ! in 
    interface  ! in :dprod
        subroutine dprod(x,y,n) ! in :dprod:dprod.f
            double precision dimension(n) :: x
            double precision :: y
            integer, optional,check(len(x)>=n),depend(x) :: n=len(x)
        end subroutine dprod
    end interface 
end python module dprod

! This file was auto-generated with f2py (version:2).
! See http://cens.ioc.ee/projects/f2py2e/
    \end{Verbatim}

    The module does not know what Fortran subroutine arguments is input and
output, so we need to manually edit the module declaration files and
mark output variables with \texttt{intent(out)} and input variable with
\texttt{intent(in)}:

    \begin{Verbatim}[commandchars=\\\{\}]
{\color{incolor}In [{\color{incolor}156}]:} \PY{o}{\PYZpc{}\PYZpc{}}\PY{k}{file} dprod.pyf
          python module dprod ! in 
              interface  ! in :dprod
                  subroutine dprod(x,y,n) ! in :dprod:dprod.f
                      double precision dimension(n), intent(in) :: x
                      double precision, intent(out) :: y
                      integer, optional,check(len(x)\PYZgt{}=n),depend(x),intent(in) :: n=len(x)
                  end subroutine dprod
              end interface 
          end python module dprod
\end{Verbatim}

    \begin{Verbatim}[commandchars=\\\{\}]
Overwriting dprod.pyf
    \end{Verbatim}

    Compile the fortran code into a module that can be included in python:

    \begin{Verbatim}[commandchars=\\\{\}]
{\color{incolor}In [{\color{incolor}157}]:} \PY{o}{!}f2py3 \PYZhy{}c dprod.pyf dprod.f
\end{Verbatim}

    \begin{Verbatim}[commandchars=\\\{\}]
running build
running config\_cc
unifing config\_cc, config, build\_clib, build\_ext, build commands --compiler options
running config\_fc
unifing config\_fc, config, build\_clib, build\_ext, build commands --fcompiler options
running build\_src
build\_src
building extension "dprod" sources
creating /tmp/tmpo0ulkq7s/src.linux-x86\_64-3.4
f2py options: []
f2py: dprod.pyf
Reading fortran codes{\ldots}
	Reading file 'dprod.pyf' (format:free)
Post-processing{\ldots}
	Block: dprod
			Block: dprod
Post-processing (stage 2){\ldots}
Building modules{\ldots}
	Building module "dprod"{\ldots}
		Constructing wrapper function "dprod"{\ldots}
		  y = dprod(x,[n])
	Wrote C/API module "dprod" to file "/tmp/tmpo0ulkq7s/src.linux-x86\_64-3.4/dprodmodule.c"
  adding '/tmp/tmpo0ulkq7s/src.linux-x86\_64-3.4/fortranobject.c' to sources.
  adding '/tmp/tmpo0ulkq7s/src.linux-x86\_64-3.4' to include\_dirs.
copying /home/dhavide/anaconda3/lib/python3.4/site-packages/numpy/f2py/src/fortranobject.c -> /tmp/tmpo0ulkq7s/src.linux-x86\_64-3.4
copying /home/dhavide/anaconda3/lib/python3.4/site-packages/numpy/f2py/src/fortranobject.h -> /tmp/tmpo0ulkq7s/src.linux-x86\_64-3.4
build\_src: building npy-pkg config files
running build\_ext
customize UnixCCompiler
customize UnixCCompiler using build\_ext
customize Gnu95FCompiler
Found executable /usr/bin/gfortran
customize Gnu95FCompiler
customize Gnu95FCompiler using build\_ext
building 'dprod' extension
compiling C sources
C compiler: gcc -pthread -DNDEBUG -g -fwrapv -O3 -Wall -Wstrict-prototypes -fPIC

creating /tmp/tmpo0ulkq7s/tmp
creating /tmp/tmpo0ulkq7s/tmp/tmpo0ulkq7s
creating /tmp/tmpo0ulkq7s/tmp/tmpo0ulkq7s/src.linux-x86\_64-3.4
compile options: '-I/tmp/tmpo0ulkq7s/src.linux-x86\_64-3.4 -I/home/dhavide/anaconda3/lib/python3.4/site-packages/numpy/core/include -I/home/dhavide/anaconda3/include/python3.4m -c'
gcc: /tmp/tmpo0ulkq7s/src.linux-x86\_64-3.4/fortranobject.c
In file included from /home/dhavide/anaconda3/lib/python3.4/site-packages/numpy/core/include/numpy/ndarraytypes.h:1804:0,
                 from /home/dhavide/anaconda3/lib/python3.4/site-packages/numpy/core/include/numpy/ndarrayobject.h:17,
                 from /home/dhavide/anaconda3/lib/python3.4/site-packages/numpy/core/include/numpy/arrayobject.h:4,
                 from /tmp/tmpo0ulkq7s/src.linux-x86\_64-3.4/fortranobject.h:13,
                 from /tmp/tmpo0ulkq7s/src.linux-x86\_64-3.4/fortranobject.c:2:
/home/dhavide/anaconda3/lib/python3.4/site-packages/numpy/core/include/numpy/npy\_1\_7\_deprecated\_api.h:15:2: warning: \#warning "Using deprecated NumPy API, disable it by " "\#defining NPY\_NO\_DEPRECATED\_API NPY\_1\_7\_API\_VERSION" [-Wcpp]
 \#warning "Using deprecated NumPy API, disable it by " \textbackslash{}
  \^{}
gcc: /tmp/tmpo0ulkq7s/src.linux-x86\_64-3.4/dprodmodule.c
In file included from /home/dhavide/anaconda3/lib/python3.4/site-packages/numpy/core/include/numpy/ndarraytypes.h:1804:0,
                 from /home/dhavide/anaconda3/lib/python3.4/site-packages/numpy/core/include/numpy/ndarrayobject.h:17,
                 from /home/dhavide/anaconda3/lib/python3.4/site-packages/numpy/core/include/numpy/arrayobject.h:4,
                 from /tmp/tmpo0ulkq7s/src.linux-x86\_64-3.4/fortranobject.h:13,
                 from /tmp/tmpo0ulkq7s/src.linux-x86\_64-3.4/dprodmodule.c:18:
/home/dhavide/anaconda3/lib/python3.4/site-packages/numpy/core/include/numpy/npy\_1\_7\_deprecated\_api.h:15:2: warning: \#warning "Using deprecated NumPy API, disable it by " "\#defining NPY\_NO\_DEPRECATED\_API NPY\_1\_7\_API\_VERSION" [-Wcpp]
 \#warning "Using deprecated NumPy API, disable it by " \textbackslash{}
  \^{}
/tmp/tmpo0ulkq7s/src.linux-x86\_64-3.4/dprodmodule.c:111:12: warning: ‘f2py\_size’ defined but not used [-Wunused-function]
 static int f2py\_size(PyArrayObject* var, {\ldots})
            \^{}
compiling Fortran sources
Fortran f77 compiler: /usr/bin/gfortran -Wall -g -ffixed-form -fno-second-underscore -fPIC -O3 -funroll-loops
Fortran f90 compiler: /usr/bin/gfortran -Wall -g -fno-second-underscore -fPIC -O3 -funroll-loops
Fortran fix compiler: /usr/bin/gfortran -Wall -g -ffixed-form -fno-second-underscore -Wall -g -fno-second-underscore -fPIC -O3 -funroll-loops
compile options: '-I/tmp/tmpo0ulkq7s/src.linux-x86\_64-3.4 -I/home/dhavide/anaconda3/lib/python3.4/site-packages/numpy/core/include -I/home/dhavide/anaconda3/include/python3.4m -c'
gfortran:f77: dprod.f
/usr/bin/gfortran -Wall -g -Wall -g -shared /tmp/tmpo0ulkq7s/tmp/tmpo0ulkq7s/src.linux-x86\_64-3.4/dprodmodule.o /tmp/tmpo0ulkq7s/tmp/tmpo0ulkq7s/src.linux-x86\_64-3.4/fortranobject.o /tmp/tmpo0ulkq7s/dprod.o -L/home/dhavide/anaconda3/lib -lpython3.4m -lgfortran -o ./dprod.cpython-34m.so
Removing build directory /tmp/tmpo0ulkq7s
    \end{Verbatim}

    \paragraph{Using the module from
Python}\label{using-the-module-from-python}

    \begin{Verbatim}[commandchars=\\\{\}]
{\color{incolor}In [{\color{incolor}158}]:} \PY{k+kn}{import} \PY{n+nn}{dprod}
\end{Verbatim}

    \begin{Verbatim}[commandchars=\\\{\}]
{\color{incolor}In [{\color{incolor}159}]:} \PY{n}{help}\PY{p}{(}\PY{n}{dprod}\PY{p}{)}
\end{Verbatim}

    \begin{Verbatim}[commandchars=\\\{\}]
Help on module dprod:

NAME
    dprod

DESCRIPTION
    This module 'dprod' is auto-generated with f2py (version:2).
    Functions:
      y = dprod(x,n=len(x))
    .

DATA
    dprod = <fortran object>

VERSION
    b'\$Revision: \$'

FILE
    /home/dhavide/repositories/scientific-python-lectures/dprod.cpython-34m.so
    \end{Verbatim}

    \begin{Verbatim}[commandchars=\\\{\}]
{\color{incolor}In [{\color{incolor}160}]:} \PY{n}{dprod}\PY{o}{.}\PY{n}{dprod}\PY{p}{(}\PY{n}{arange}\PY{p}{(}\PY{l+m+mi}{1}\PY{p}{,}\PY{l+m+mi}{50}\PY{p}{)}\PY{p}{)}
\end{Verbatim}

            \begin{Verbatim}[commandchars=\\\{\}]
{\color{outcolor}Out[{\color{outcolor}160}]:} 6.082818640342675e+62
\end{Verbatim}
        
    \begin{Verbatim}[commandchars=\\\{\}]
{\color{incolor}In [{\color{incolor}161}]:} \PY{c}{\PYZsh{} compare to numpy}
          \PY{n}{prod}\PY{p}{(}\PY{n}{arange}\PY{p}{(}\PY{l+m+mf}{1.0}\PY{p}{,}\PY{l+m+mf}{50.0}\PY{p}{)}\PY{p}{)}
\end{Verbatim}

            \begin{Verbatim}[commandchars=\\\{\}]
{\color{outcolor}Out[{\color{outcolor}161}]:} 6.0828186403426752e+62
\end{Verbatim}
        
    \begin{Verbatim}[commandchars=\\\{\}]
{\color{incolor}In [{\color{incolor}162}]:} \PY{n}{dprod}\PY{o}{.}\PY{n}{dprod}\PY{p}{(}\PY{n}{arange}\PY{p}{(}\PY{l+m+mi}{1}\PY{p}{,}\PY{l+m+mi}{10}\PY{p}{)}\PY{p}{,} \PY{l+m+mi}{5}\PY{p}{)} \PY{c}{\PYZsh{} only the 5 first elements}
\end{Verbatim}

            \begin{Verbatim}[commandchars=\\\{\}]
{\color{outcolor}Out[{\color{outcolor}162}]:} 120.0
\end{Verbatim}
        
    Compare performance:

    \begin{Verbatim}[commandchars=\\\{\}]
{\color{incolor}In [{\color{incolor}163}]:} \PY{n}{xvec} \PY{o}{=} \PY{n}{rand}\PY{p}{(}\PY{l+m+mi}{500}\PY{p}{)}
\end{Verbatim}

    \begin{Verbatim}[commandchars=\\\{\}]
{\color{incolor}In [{\color{incolor}164}]:} \PY{n}{timeit} \PY{n}{dprod}\PY{o}{.}\PY{n}{dprod}\PY{p}{(}\PY{n}{xvec}\PY{p}{)}
\end{Verbatim}

    \begin{Verbatim}[commandchars=\\\{\}]
The slowest run took 5.61 times longer than the fastest. This could mean that an intermediate result is being cached 
1000000 loops, best of 3: 1.63 µs per loop
    \end{Verbatim}

    \begin{Verbatim}[commandchars=\\\{\}]
{\color{incolor}In [{\color{incolor}165}]:} \PY{n}{timeit} \PY{n}{xvec}\PY{o}{.}\PY{n}{prod}\PY{p}{(}\PY{p}{)}
\end{Verbatim}

    \begin{Verbatim}[commandchars=\\\{\}]
The slowest run took 6.64 times longer than the fastest. This could mean that an intermediate result is being cached 
100000 loops, best of 3: 8.46 µs per loop
    \end{Verbatim}

    \subsubsection{Example 2: cumulative sum, vector input and vector
output}\label{example-2-cumulative-sum-vector-input-and-vector-output}

    The cummulative sum function for an array of data is a good example of a
loop intense algorithm: Loop through a vector and store the cummulative
sum in another vector.

    \begin{Verbatim}[commandchars=\\\{\}]
{\color{incolor}In [{\color{incolor}166}]:} \PY{c}{\PYZsh{} simple python algorithm: example of a SLOW implementation}
          \PY{c}{\PYZsh{} Why? Because the loop is implemented in python.}
          \PY{k}{def} \PY{n+nf}{py\PYZus{}dcumsum}\PY{p}{(}\PY{n}{a}\PY{p}{)}\PY{p}{:}
              \PY{n}{b} \PY{o}{=} \PY{n}{empty\PYZus{}like}\PY{p}{(}\PY{n}{a}\PY{p}{)}
              \PY{n}{b}\PY{p}{[}\PY{l+m+mi}{0}\PY{p}{]} \PY{o}{=} \PY{n}{a}\PY{p}{[}\PY{l+m+mi}{0}\PY{p}{]}
              \PY{k}{for} \PY{n}{n} \PY{o+ow}{in} \PY{n+nb}{range}\PY{p}{(}\PY{l+m+mi}{1}\PY{p}{,}\PY{n+nb}{len}\PY{p}{(}\PY{n}{a}\PY{p}{)}\PY{p}{)}\PY{p}{:}
                  \PY{n}{b}\PY{p}{[}\PY{n}{n}\PY{p}{]} \PY{o}{=} \PY{n}{b}\PY{p}{[}\PY{n}{n}\PY{o}{\PYZhy{}}\PY{l+m+mi}{1}\PY{p}{]}\PY{o}{+}\PY{n}{a}\PY{p}{[}\PY{n}{n}\PY{p}{]}
              \PY{k}{return} \PY{n}{b}
\end{Verbatim}

    Fortran subroutine for the same thing: here we have added the
\texttt{intent(in)} and \texttt{intent(out)} as comment lines in the
original fortran code, so we do not need to manually edit the fortran
module declaration file generated by \texttt{f2py}.

    \begin{Verbatim}[commandchars=\\\{\}]
{\color{incolor}In [{\color{incolor}167}]:} \PY{o}{\PYZpc{}\PYZpc{}}\PY{k}{file} dcumsum.f
          c File dcumsum.f
                 subroutine dcumsum(a, b, n)
                 double precision a(n)
                 double precision b(n)
                 integer n
          cf2py  intent(in) :: a
          cf2py  intent(out) :: b
          cf2py  intent(hide) :: n
          
                 b(1) = a(1)
                 do 100 i=2, n
                     b(i) = b(i\PYZhy{}1) + a(i)
          100    continue
                 end
\end{Verbatim}

    \begin{Verbatim}[commandchars=\\\{\}]
Overwriting dcumsum.f
    \end{Verbatim}

    We can directly compile the fortran code to a python module:

    \begin{Verbatim}[commandchars=\\\{\}]
{\color{incolor}In [{\color{incolor}168}]:} \PY{o}{!}f2py3 \PYZhy{}c dcumsum.f \PYZhy{}m dcumsum
\end{Verbatim}

    \begin{Verbatim}[commandchars=\\\{\}]
running build
running config\_cc
unifing config\_cc, config, build\_clib, build\_ext, build commands --compiler options
running config\_fc
unifing config\_fc, config, build\_clib, build\_ext, build commands --fcompiler options
running build\_src
build\_src
building extension "dcumsum" sources
f2py options: []
f2py:> /tmp/tmpe46xtmge/src.linux-x86\_64-3.4/dcumsummodule.c
creating /tmp/tmpe46xtmge/src.linux-x86\_64-3.4
Reading fortran codes{\ldots}
	Reading file 'dcumsum.f' (format:fix,strict)
Post-processing{\ldots}
	Block: dcumsum
			Block: dcumsum
Post-processing (stage 2){\ldots}
Building modules{\ldots}
	Building module "dcumsum"{\ldots}
		Constructing wrapper function "dcumsum"{\ldots}
		  b = dcumsum(a)
	Wrote C/API module "dcumsum" to file "/tmp/tmpe46xtmge/src.linux-x86\_64-3.4/dcumsummodule.c"
  adding '/tmp/tmpe46xtmge/src.linux-x86\_64-3.4/fortranobject.c' to sources.
  adding '/tmp/tmpe46xtmge/src.linux-x86\_64-3.4' to include\_dirs.
copying /home/dhavide/anaconda3/lib/python3.4/site-packages/numpy/f2py/src/fortranobject.c -> /tmp/tmpe46xtmge/src.linux-x86\_64-3.4
copying /home/dhavide/anaconda3/lib/python3.4/site-packages/numpy/f2py/src/fortranobject.h -> /tmp/tmpe46xtmge/src.linux-x86\_64-3.4
build\_src: building npy-pkg config files
running build\_ext
customize UnixCCompiler
customize UnixCCompiler using build\_ext
customize Gnu95FCompiler
Found executable /usr/bin/gfortran
customize Gnu95FCompiler
customize Gnu95FCompiler using build\_ext
building 'dcumsum' extension
compiling C sources
C compiler: gcc -pthread -DNDEBUG -g -fwrapv -O3 -Wall -Wstrict-prototypes -fPIC

creating /tmp/tmpe46xtmge/tmp
creating /tmp/tmpe46xtmge/tmp/tmpe46xtmge
creating /tmp/tmpe46xtmge/tmp/tmpe46xtmge/src.linux-x86\_64-3.4
compile options: '-I/tmp/tmpe46xtmge/src.linux-x86\_64-3.4 -I/home/dhavide/anaconda3/lib/python3.4/site-packages/numpy/core/include -I/home/dhavide/anaconda3/include/python3.4m -c'
gcc: /tmp/tmpe46xtmge/src.linux-x86\_64-3.4/dcumsummodule.c
In file included from /home/dhavide/anaconda3/lib/python3.4/site-packages/numpy/core/include/numpy/ndarraytypes.h:1804:0,
                 from /home/dhavide/anaconda3/lib/python3.4/site-packages/numpy/core/include/numpy/ndarrayobject.h:17,
                 from /home/dhavide/anaconda3/lib/python3.4/site-packages/numpy/core/include/numpy/arrayobject.h:4,
                 from /tmp/tmpe46xtmge/src.linux-x86\_64-3.4/fortranobject.h:13,
                 from /tmp/tmpe46xtmge/src.linux-x86\_64-3.4/dcumsummodule.c:18:
/home/dhavide/anaconda3/lib/python3.4/site-packages/numpy/core/include/numpy/npy\_1\_7\_deprecated\_api.h:15:2: warning: \#warning "Using deprecated NumPy API, disable it by " "\#defining NPY\_NO\_DEPRECATED\_API NPY\_1\_7\_API\_VERSION" [-Wcpp]
 \#warning "Using deprecated NumPy API, disable it by " \textbackslash{}
  \^{}
/tmp/tmpe46xtmge/src.linux-x86\_64-3.4/dcumsummodule.c:111:12: warning: ‘f2py\_size’ defined but not used [-Wunused-function]
 static int f2py\_size(PyArrayObject* var, {\ldots})
            \^{}
gcc: /tmp/tmpe46xtmge/src.linux-x86\_64-3.4/fortranobject.c
In file included from /home/dhavide/anaconda3/lib/python3.4/site-packages/numpy/core/include/numpy/ndarraytypes.h:1804:0,
                 from /home/dhavide/anaconda3/lib/python3.4/site-packages/numpy/core/include/numpy/ndarrayobject.h:17,
                 from /home/dhavide/anaconda3/lib/python3.4/site-packages/numpy/core/include/numpy/arrayobject.h:4,
                 from /tmp/tmpe46xtmge/src.linux-x86\_64-3.4/fortranobject.h:13,
                 from /tmp/tmpe46xtmge/src.linux-x86\_64-3.4/fortranobject.c:2:
/home/dhavide/anaconda3/lib/python3.4/site-packages/numpy/core/include/numpy/npy\_1\_7\_deprecated\_api.h:15:2: warning: \#warning "Using deprecated NumPy API, disable it by " "\#defining NPY\_NO\_DEPRECATED\_API NPY\_1\_7\_API\_VERSION" [-Wcpp]
 \#warning "Using deprecated NumPy API, disable it by " \textbackslash{}
  \^{}
compiling Fortran sources
Fortran f77 compiler: /usr/bin/gfortran -Wall -g -ffixed-form -fno-second-underscore -fPIC -O3 -funroll-loops
Fortran f90 compiler: /usr/bin/gfortran -Wall -g -fno-second-underscore -fPIC -O3 -funroll-loops
Fortran fix compiler: /usr/bin/gfortran -Wall -g -ffixed-form -fno-second-underscore -Wall -g -fno-second-underscore -fPIC -O3 -funroll-loops
compile options: '-I/tmp/tmpe46xtmge/src.linux-x86\_64-3.4 -I/home/dhavide/anaconda3/lib/python3.4/site-packages/numpy/core/include -I/home/dhavide/anaconda3/include/python3.4m -c'
gfortran:f77: dcumsum.f
/usr/bin/gfortran -Wall -g -Wall -g -shared /tmp/tmpe46xtmge/tmp/tmpe46xtmge/src.linux-x86\_64-3.4/dcumsummodule.o /tmp/tmpe46xtmge/tmp/tmpe46xtmge/src.linux-x86\_64-3.4/fortranobject.o /tmp/tmpe46xtmge/dcumsum.o -L/home/dhavide/anaconda3/lib -lpython3.4m -lgfortran -o ./dcumsum.cpython-34m.so
Removing build directory /tmp/tmpe46xtmge
    \end{Verbatim}

    \begin{Verbatim}[commandchars=\\\{\}]
{\color{incolor}In [{\color{incolor}169}]:} \PY{k+kn}{import} \PY{n+nn}{dcumsum}
\end{Verbatim}

    \begin{Verbatim}[commandchars=\\\{\}]
{\color{incolor}In [{\color{incolor}170}]:} \PY{n}{a} \PY{o}{=} \PY{n}{array}\PY{p}{(}\PY{p}{[}\PY{l+m+mf}{1.0}\PY{p}{,}\PY{l+m+mf}{2.0}\PY{p}{,}\PY{l+m+mf}{3.0}\PY{p}{,}\PY{l+m+mf}{4.0}\PY{p}{,}\PY{l+m+mf}{5.0}\PY{p}{,}\PY{l+m+mf}{6.0}\PY{p}{,}\PY{l+m+mf}{7.0}\PY{p}{,}\PY{l+m+mf}{8.0}\PY{p}{]}\PY{p}{)}
\end{Verbatim}

    \begin{Verbatim}[commandchars=\\\{\}]
{\color{incolor}In [{\color{incolor}171}]:} \PY{n}{py\PYZus{}dcumsum}\PY{p}{(}\PY{n}{a}\PY{p}{)}
\end{Verbatim}

            \begin{Verbatim}[commandchars=\\\{\}]
{\color{outcolor}Out[{\color{outcolor}171}]:} array([  1.,   3.,   6.,  10.,  15.,  21.,  28.,  36.])
\end{Verbatim}
        
    \begin{Verbatim}[commandchars=\\\{\}]
{\color{incolor}In [{\color{incolor}172}]:} \PY{n}{dcumsum}\PY{o}{.}\PY{n}{dcumsum}\PY{p}{(}\PY{n}{a}\PY{p}{)}
\end{Verbatim}

            \begin{Verbatim}[commandchars=\\\{\}]
{\color{outcolor}Out[{\color{outcolor}172}]:} array([  1.,   3.,   6.,  10.,  15.,  21.,  28.,  36.])
\end{Verbatim}
        
    \begin{Verbatim}[commandchars=\\\{\}]
{\color{incolor}In [{\color{incolor}173}]:} \PY{n}{cumsum}\PY{p}{(}\PY{n}{a}\PY{p}{)}
\end{Verbatim}

            \begin{Verbatim}[commandchars=\\\{\}]
{\color{outcolor}Out[{\color{outcolor}173}]:} array([  1.,   3.,   6.,  10.,  15.,  21.,  28.,  36.])
\end{Verbatim}
        
    Benchmark the different implementations:

    \begin{Verbatim}[commandchars=\\\{\}]
{\color{incolor}In [{\color{incolor}174}]:} \PY{n}{a} \PY{o}{=} \PY{n}{rand}\PY{p}{(}\PY{l+m+mi}{10000}\PY{p}{)}
\end{Verbatim}

    \begin{Verbatim}[commandchars=\\\{\}]
{\color{incolor}In [{\color{incolor}175}]:} \PY{n}{timeit} \PY{n}{py\PYZus{}dcumsum}\PY{p}{(}\PY{n}{a}\PY{p}{)}
\end{Verbatim}

    \begin{Verbatim}[commandchars=\\\{\}]
100 loops, best of 3: 5.41 ms per loop
    \end{Verbatim}

    \begin{Verbatim}[commandchars=\\\{\}]
{\color{incolor}In [{\color{incolor}176}]:} \PY{n}{timeit} \PY{n}{dcumsum}\PY{o}{.}\PY{n}{dcumsum}\PY{p}{(}\PY{n}{a}\PY{p}{)}
\end{Verbatim}

    \begin{Verbatim}[commandchars=\\\{\}]
10000 loops, best of 3: 18.5 µs per loop
    \end{Verbatim}

    \begin{Verbatim}[commandchars=\\\{\}]
{\color{incolor}In [{\color{incolor}177}]:} \PY{n}{timeit} \PY{n}{a}\PY{o}{.}\PY{n}{cumsum}\PY{p}{(}\PY{p}{)}
\end{Verbatim}

    \begin{Verbatim}[commandchars=\\\{\}]
The slowest run took 17.25 times longer than the fastest. This could mean that an intermediate result is being cached 
10000 loops, best of 3: 47 µs per loop
    \end{Verbatim}

    \subsubsection{Further reading}\label{further-reading}

    \begin{enumerate}
\def\labelenumi{\arabic{enumi}.}
\itemsep1pt\parskip0pt\parsep0pt
\item
  http://www.scipy.org/F2py
\item
  http://dsnra.jpl.nasa.gov/software/Python/F2PY\_tutorial.pdf
\item
  http://www.shocksolution.com/2009/09/f2py-binding-fortran-python/
\end{enumerate}

    \subsection{C}\label{c}

    \subsection{ctypes}\label{ctypes}

    ctypes is a Python library for calling out to C code. It is not as
automatic as \texttt{f2py}, and we manually need to load the library and
set properties such as the functions return and argument types. On the
otherhand we do not need to touch the C code at all.

    \begin{Verbatim}[commandchars=\\\{\}]
{\color{incolor}In [{\color{incolor}178}]:} \PY{o}{\PYZpc{}\PYZpc{}}\PY{k}{file} functions.c
          
          \PYZsh{}include \PYZlt{}stdio.h\PYZgt{}
          
          void hello(int n);
          
          double dprod(double *x, int n);
          
          void dcumsum(double *a, double *b, int n);
          
          void
          hello(int n)
          \PYZob{}
              int i;
              
              for (i = 0; i \PYZlt{} n; i++)
              \PYZob{}
                  printf(\PYZdq{}C says hello\PYZbs{}n\PYZdq{});
              \PYZcb{}
          \PYZcb{}
          
          
          double 
          dprod(double *x, int n)
          \PYZob{}
              int i;
              double y = 1.0;
              
              for (i = 0; i \PYZlt{} n; i++)
              \PYZob{}
                  y *= x[i];
              \PYZcb{}
          
              return y;
          \PYZcb{}
          
          void
          dcumsum(double *a, double *b, int n)
          \PYZob{}
              int i;
              
              b[0] = a[0];
              for (i = 1; i \PYZlt{} n; i++)
              \PYZob{}
                  b[i] = a[i] + b[i\PYZhy{}1];
              \PYZcb{}
          \PYZcb{}
\end{Verbatim}

    \begin{Verbatim}[commandchars=\\\{\}]
Overwriting functions.c
    \end{Verbatim}

    Compile the C file into a shared library:

    \begin{Verbatim}[commandchars=\\\{\}]
{\color{incolor}In [{\color{incolor}179}]:} \PY{o}{!}gcc \PYZhy{}c \PYZhy{}Wall \PYZhy{}O2 \PYZhy{}Wall \PYZhy{}ansi \PYZhy{}pedantic \PYZhy{}fPIC \PYZhy{}o functions.o functions.c
          \PY{o}{!}gcc \PYZhy{}o libfunctions.so \PYZhy{}shared functions.o
\end{Verbatim}

    The result is a compiled shared library \texttt{libfunctions.so}:

    \begin{Verbatim}[commandchars=\\\{\}]
{\color{incolor}In [{\color{incolor}180}]:} \PY{o}{!}file libfunctions.so
\end{Verbatim}

    \begin{Verbatim}[commandchars=\\\{\}]
libfunctions.so: ELF 64-bit LSB  shared object, x86-64, version 1 (SYSV), dynamically linked, BuildID[sha1]=8e644d6e751685aa836f078d4caf1734f3d01eb4, not stripped
    \end{Verbatim}

    Now we need to write wrapper functions to access the C library: To load
the library we use the ctypes package, which included in the Python
standard library (with extensions from numpy for passing arrays to C).
Then we manually set the types of the argument and return values (no
automatic code inspection here!).

    \begin{Verbatim}[commandchars=\\\{\}]
{\color{incolor}In [{\color{incolor}181}]:} \PY{o}{\PYZpc{}\PYZpc{}}\PY{k}{file} functions.py
          
          import numpy
          import ctypes
          
          \PYZus{}libfunctions = numpy.ctypeslib.load\PYZus{}library(\PYZsq{}libfunctions\PYZsq{}, \PYZsq{}.\PYZsq{})
          
          \PYZus{}libfunctions.hello.argtypes = [ctypes.c\PYZus{}int]
          \PYZus{}libfunctions.hello.restype  =  ctypes.c\PYZus{}void\PYZus{}p
          
          \PYZus{}libfunctions.dprod.argtypes = [numpy.ctypeslib.ndpointer(dtype=numpy.float), ctypes.c\PYZus{}int]
          \PYZus{}libfunctions.dprod.restype  = ctypes.c\PYZus{}double
          
          \PYZus{}libfunctions.dcumsum.argtypes = [numpy.ctypeslib.ndpointer(dtype=numpy.float), numpy.ctypeslib.ndpointer(dtype=numpy.float), ctypes.c\PYZus{}int]
          \PYZus{}libfunctions.dcumsum.restype  = ctypes.c\PYZus{}void\PYZus{}p
          
          def hello(n):
              return \PYZus{}libfunctions.hello(int(n))
          
          def dprod(x, n=None):
              if n is None:
                  n = len(x)
              x = numpy.asarray(x, dtype=numpy.float)
              return \PYZus{}libfunctions.dprod(x, int(n))
          
          def dcumsum(a, n):
              a = numpy.asarray(a, dtype=numpy.float)
              b = numpy.empty(len(a), dtype=numpy.float)
              \PYZus{}libfunctions.dcumsum(a, b, int(n))
              return b
\end{Verbatim}

    \begin{Verbatim}[commandchars=\\\{\}]
Overwriting functions.py
    \end{Verbatim}

    \begin{Verbatim}[commandchars=\\\{\}]
{\color{incolor}In [{\color{incolor}182}]:} \PY{o}{\PYZpc{}\PYZpc{}}\PY{k}{file} run\PYZus{}hello\PYZus{}c.py
          
          import functions
          
          functions.hello(3)
\end{Verbatim}

    \begin{Verbatim}[commandchars=\\\{\}]
Overwriting run\_hello\_c.py
    \end{Verbatim}

    \begin{Verbatim}[commandchars=\\\{\}]
{\color{incolor}In [{\color{incolor}183}]:} \PY{o}{!}python run\PYZus{}hello\PYZus{}c.py
\end{Verbatim}

    \begin{Verbatim}[commandchars=\\\{\}]
C says hello
C says hello
C says hello
    \end{Verbatim}

    \begin{Verbatim}[commandchars=\\\{\}]
{\color{incolor}In [{\color{incolor}184}]:} \PY{k+kn}{import} \PY{n+nn}{functions}
\end{Verbatim}

    \subsubsection{Product function:}\label{product-function}

    \begin{Verbatim}[commandchars=\\\{\}]
{\color{incolor}In [{\color{incolor}185}]:} \PY{n}{functions}\PY{o}{.}\PY{n}{dprod}\PY{p}{(}\PY{p}{[}\PY{l+m+mi}{1}\PY{p}{,}\PY{l+m+mi}{2}\PY{p}{,}\PY{l+m+mi}{3}\PY{p}{,}\PY{l+m+mi}{4}\PY{p}{,}\PY{l+m+mi}{5}\PY{p}{]}\PY{p}{)} 
\end{Verbatim}

            \begin{Verbatim}[commandchars=\\\{\}]
{\color{outcolor}Out[{\color{outcolor}185}]:} 120.0
\end{Verbatim}
        
    \subsubsection{Cumulative sum:}\label{cumulative-sum}

    \begin{Verbatim}[commandchars=\\\{\}]
{\color{incolor}In [{\color{incolor}186}]:} \PY{n}{a} \PY{o}{=} \PY{n}{rand}\PY{p}{(}\PY{l+m+mi}{100000}\PY{p}{)}
\end{Verbatim}

    \begin{Verbatim}[commandchars=\\\{\}]
{\color{incolor}In [{\color{incolor}187}]:} \PY{n}{res\PYZus{}c} \PY{o}{=} \PY{n}{functions}\PY{o}{.}\PY{n}{dcumsum}\PY{p}{(}\PY{n}{a}\PY{p}{,} \PY{n+nb}{len}\PY{p}{(}\PY{n}{a}\PY{p}{)}\PY{p}{)} 
\end{Verbatim}

    \begin{Verbatim}[commandchars=\\\{\}]
{\color{incolor}In [{\color{incolor}188}]:} \PY{n}{res\PYZus{}fortran} \PY{o}{=} \PY{n}{dcumsum}\PY{o}{.}\PY{n}{dcumsum}\PY{p}{(}\PY{n}{a}\PY{p}{)}
\end{Verbatim}

    \begin{Verbatim}[commandchars=\\\{\}]
{\color{incolor}In [{\color{incolor}189}]:} \PY{n}{res\PYZus{}c} \PY{o}{\PYZhy{}} \PY{n}{res\PYZus{}fortran}
\end{Verbatim}

            \begin{Verbatim}[commandchars=\\\{\}]
{\color{outcolor}Out[{\color{outcolor}189}]:} array([ 0.,  0.,  0., {\ldots},  0.,  0.,  0.])
\end{Verbatim}
        
    \subsubsection{Simple benchmark}\label{simple-benchmark}

    \begin{Verbatim}[commandchars=\\\{\}]
{\color{incolor}In [{\color{incolor}190}]:} \PY{n}{timeit} \PY{n}{functions}\PY{o}{.}\PY{n}{dcumsum}\PY{p}{(}\PY{n}{a}\PY{p}{,} \PY{n+nb}{len}\PY{p}{(}\PY{n}{a}\PY{p}{)}\PY{p}{)}
\end{Verbatim}

    \begin{Verbatim}[commandchars=\\\{\}]
1000 loops, best of 3: 517 µs per loop
    \end{Verbatim}

    \begin{Verbatim}[commandchars=\\\{\}]
{\color{incolor}In [{\color{incolor}191}]:} \PY{n}{timeit} \PY{n}{dcumsum}\PY{o}{.}\PY{n}{dcumsum}\PY{p}{(}\PY{n}{a}\PY{p}{)}
\end{Verbatim}

    \begin{Verbatim}[commandchars=\\\{\}]
1000 loops, best of 3: 247 µs per loop
    \end{Verbatim}

    \begin{Verbatim}[commandchars=\\\{\}]
{\color{incolor}In [{\color{incolor}192}]:} \PY{n}{timeit} \PY{n}{a}\PY{o}{.}\PY{n}{cumsum}\PY{p}{(}\PY{p}{)}
\end{Verbatim}

    \begin{Verbatim}[commandchars=\\\{\}]
1000 loops, best of 3: 517 µs per loop
    \end{Verbatim}

    \subsubsection{Further reading}\label{further-reading}

    \begin{itemize}
\itemsep1pt\parskip0pt\parsep0pt
\item
  http://docs.python.org/2/library/ctypes.html
\item
  http://www.scipy.org/Cookbook/Ctypes
\end{itemize}

    \subsection{Cython}\label{cython}

    A hybrid between python and C that can be compiled: Basically Python
code with type declarations.

    \begin{Verbatim}[commandchars=\\\{\}]
{\color{incolor}In [{\color{incolor}193}]:} \PY{o}{\PYZpc{}\PYZpc{}}\PY{k}{file} cy\PYZus{}dcumsum.pyx
          
          cimport numpy
          
          def dcumsum(numpy.ndarray[numpy.float64\PYZus{}t, ndim=1] a, numpy.ndarray[numpy.float64\PYZus{}t, ndim=1] b):
              cdef int i, n = len(a)
              b[0] = a[0]
              for i from 1 \PYZlt{}= i \PYZlt{} n:
                  b[i] = b[i\PYZhy{}1] + a[i]
              return b
\end{Verbatim}

    \begin{Verbatim}[commandchars=\\\{\}]
Overwriting cy\_dcumsum.pyx
    \end{Verbatim}

    A build file for generating C code and compiling it into a Python
module.

    \begin{Verbatim}[commandchars=\\\{\}]
{\color{incolor}In [{\color{incolor}194}]:} \PY{o}{\PYZpc{}\PYZpc{}}\PY{k}{file} setup.py
          import numpy
          from distutils.core import setup
          from distutils.extension import Extension
          from Cython.Distutils import build\PYZus{}ext
          
          setup(
              cmdclass = \PYZob{}\PYZsq{}build\PYZus{}ext\PYZsq{}: build\PYZus{}ext\PYZcb{},
              ext\PYZus{}modules = [Extension(\PYZdq{}cy\PYZus{}dcumsum\PYZdq{}, [\PYZdq{}cy\PYZus{}dcumsum.pyx\PYZdq{}], include\PYZus{}dirs=[numpy.get\PYZus{}include()]),
              ],
          )
\end{Verbatim}

    \begin{Verbatim}[commandchars=\\\{\}]
Overwriting setup.py
    \end{Verbatim}

    \begin{Verbatim}[commandchars=\\\{\}]
{\color{incolor}In [{\color{incolor}195}]:} \PY{o}{!}python setup.py build\PYZus{}ext \PYZhy{}\PYZhy{}inplace
\end{Verbatim}

    \begin{Verbatim}[commandchars=\\\{\}]
running build\_ext
cythoning cy\_dcumsum.pyx to cy\_dcumsum.c
building 'cy\_dcumsum' extension
gcc -pthread -DNDEBUG -g -fwrapv -O3 -Wall -Wstrict-prototypes -fPIC -I/home/dhavide/anaconda3/lib/python3.4/site-packages/numpy/core/include -I/home/dhavide/anaconda3/include/python3.4m -c cy\_dcumsum.c -o build/temp.linux-x86\_64-3.4/cy\_dcumsum.o
In file included from /home/dhavide/anaconda3/lib/python3.4/site-packages/numpy/core/include/numpy/ndarraytypes.h:1804:0,
                 from /home/dhavide/anaconda3/lib/python3.4/site-packages/numpy/core/include/numpy/ndarrayobject.h:17,
                 from /home/dhavide/anaconda3/lib/python3.4/site-packages/numpy/core/include/numpy/arrayobject.h:4,
                 from cy\_dcumsum.c:257:
/home/dhavide/anaconda3/lib/python3.4/site-packages/numpy/core/include/numpy/npy\_1\_7\_deprecated\_api.h:15:2: warning: \#warning "Using deprecated NumPy API, disable it by " "\#defining NPY\_NO\_DEPRECATED\_API NPY\_1\_7\_API\_VERSION" [-Wcpp]
 \#warning "Using deprecated NumPy API, disable it by " \textbackslash{}
  \^{}
In file included from /home/dhavide/anaconda3/lib/python3.4/site-packages/numpy/core/include/numpy/ndarrayobject.h:26:0,
                 from /home/dhavide/anaconda3/lib/python3.4/site-packages/numpy/core/include/numpy/arrayobject.h:4,
                 from cy\_dcumsum.c:257:
/home/dhavide/anaconda3/lib/python3.4/site-packages/numpy/core/include/numpy/\_\_multiarray\_api.h:1629:1: warning: ‘\_import\_array’ defined but not used [-Wunused-function]
 \_import\_array(void)
 \^{}
In file included from /home/dhavide/anaconda3/lib/python3.4/site-packages/numpy/core/include/numpy/ufuncobject.h:317:0,
                 from cy\_dcumsum.c:258:
/home/dhavide/anaconda3/lib/python3.4/site-packages/numpy/core/include/numpy/\_\_ufunc\_api.h:241:1: warning: ‘\_import\_umath’ defined but not used [-Wunused-function]
 \_import\_umath(void)
 \^{}
gcc -pthread -shared -L/home/dhavide/anaconda3/lib -Wl,-rpath=/home/dhavide/anaconda3/lib,--no-as-needed build/temp.linux-x86\_64-3.4/cy\_dcumsum.o -L/home/dhavide/anaconda3/lib -lpython3.4m -o /home/dhavide/repositories/scientific-python-lectures/cy\_dcumsum.cpython-34m.so
    \end{Verbatim}

    \begin{Verbatim}[commandchars=\\\{\}]
{\color{incolor}In [{\color{incolor}196}]:} \PY{k+kn}{import} \PY{n+nn}{cy\PYZus{}dcumsum}
\end{Verbatim}

    \begin{Verbatim}[commandchars=\\\{\}]
{\color{incolor}In [{\color{incolor}197}]:} \PY{n}{a} \PY{o}{=} \PY{n}{array}\PY{p}{(}\PY{p}{[}\PY{l+m+mi}{1}\PY{p}{,}\PY{l+m+mi}{2}\PY{p}{,}\PY{l+m+mi}{3}\PY{p}{,}\PY{l+m+mi}{4}\PY{p}{]}\PY{p}{,} \PY{n}{dtype}\PY{o}{=}\PY{n+nb}{float}\PY{p}{)}
          \PY{n}{b} \PY{o}{=} \PY{n}{empty\PYZus{}like}\PY{p}{(}\PY{n}{a}\PY{p}{)}
          \PY{n}{cy\PYZus{}dcumsum}\PY{o}{.}\PY{n}{dcumsum}\PY{p}{(}\PY{n}{a}\PY{p}{,}\PY{n}{b}\PY{p}{)}
          \PY{n}{b}
\end{Verbatim}

            \begin{Verbatim}[commandchars=\\\{\}]
{\color{outcolor}Out[{\color{outcolor}197}]:} array([  1.,   3.,   6.,  10.])
\end{Verbatim}
        
    \begin{Verbatim}[commandchars=\\\{\}]
{\color{incolor}In [{\color{incolor}198}]:} \PY{n}{a} \PY{o}{=} \PY{n}{array}\PY{p}{(}\PY{p}{[}\PY{l+m+mf}{1.0}\PY{p}{,} \PY{l+m+mf}{2.0}\PY{p}{,} \PY{l+m+mf}{3.0}\PY{p}{,} \PY{l+m+mf}{4.0}\PY{p}{,} \PY{l+m+mf}{5.0}\PY{p}{,} \PY{l+m+mf}{6.0}\PY{p}{,} \PY{l+m+mf}{7.0}\PY{p}{,} \PY{l+m+mf}{8.0}\PY{p}{]}\PY{p}{)}
\end{Verbatim}

    \begin{Verbatim}[commandchars=\\\{\}]
{\color{incolor}In [{\color{incolor}199}]:} \PY{n}{b} \PY{o}{=} \PY{n}{empty\PYZus{}like}\PY{p}{(}\PY{n}{a}\PY{p}{)}
          \PY{n}{cy\PYZus{}dcumsum}\PY{o}{.}\PY{n}{dcumsum}\PY{p}{(}\PY{n}{a}\PY{p}{,} \PY{n}{b}\PY{p}{)}
          \PY{n}{b}
\end{Verbatim}

            \begin{Verbatim}[commandchars=\\\{\}]
{\color{outcolor}Out[{\color{outcolor}199}]:} array([  1.,   3.,   6.,  10.,  15.,  21.,  28.,  36.])
\end{Verbatim}
        
    \begin{Verbatim}[commandchars=\\\{\}]
{\color{incolor}In [{\color{incolor}200}]:} \PY{n}{py\PYZus{}dcumsum}\PY{p}{(}\PY{n}{a}\PY{p}{)}
\end{Verbatim}

            \begin{Verbatim}[commandchars=\\\{\}]
{\color{outcolor}Out[{\color{outcolor}200}]:} array([  1.,   3.,   6.,  10.,  15.,  21.,  28.,  36.])
\end{Verbatim}
        
    \begin{Verbatim}[commandchars=\\\{\}]
{\color{incolor}In [{\color{incolor}201}]:} \PY{n}{a} \PY{o}{=} \PY{n}{rand}\PY{p}{(}\PY{l+m+mi}{100000}\PY{p}{)}
          \PY{n}{b} \PY{o}{=} \PY{n}{empty\PYZus{}like}\PY{p}{(}\PY{n}{a}\PY{p}{)}
\end{Verbatim}

    \begin{Verbatim}[commandchars=\\\{\}]
{\color{incolor}In [{\color{incolor}202}]:} \PY{n}{timeit} \PY{n}{py\PYZus{}dcumsum}\PY{p}{(}\PY{n}{a}\PY{p}{)}
\end{Verbatim}

    \begin{Verbatim}[commandchars=\\\{\}]
10 loops, best of 3: 72.7 ms per loop
    \end{Verbatim}

    \begin{Verbatim}[commandchars=\\\{\}]
{\color{incolor}In [{\color{incolor}203}]:} \PY{n}{timeit} \PY{n}{cy\PYZus{}dcumsum}\PY{o}{.}\PY{n}{dcumsum}\PY{p}{(}\PY{n}{a}\PY{p}{,}\PY{n}{b}\PY{p}{)}
\end{Verbatim}

    \begin{Verbatim}[commandchars=\\\{\}]
1000 loops, best of 3: 469 µs per loop
    \end{Verbatim}

    \subsubsection{Cython in the IPython
notebook}\label{cython-in-the-ipython-notebook}

    When working with the IPython (especially in the notebook), there is a
more convenient way of compiling and loading Cython code. Using the
\texttt{\%\%cython} IPython magic (command to IPython), we can simply
type the Cython code in a code cell and let IPython take care of the
conversion to C code, compilation and loading of the function. To be
able to use the \texttt{\%\%cython} magic, we first need to load the
extension \texttt{cythonmagic}:

    \begin{Verbatim}[commandchars=\\\{\}]
{\color{incolor}In [{\color{incolor}204}]:} \PY{o}{\PYZpc{}}\PY{k}{load\PYZus{}ext} Cython
\end{Verbatim}

    \begin{Verbatim}[commandchars=\\\{\}]
The Cython extension is already loaded. To reload it, use:
  \%reload\_ext Cython
    \end{Verbatim}

    \begin{Verbatim}[commandchars=\\\{\}]
{\color{incolor}In [{\color{incolor}205}]:} \PY{o}{\PYZpc{}\PYZpc{}}\PY{k}{cython}
          
          cimport numpy
          
          def cy\PYZus{}dcumsum2(numpy.ndarray[numpy.float64\PYZus{}t, ndim=1] a, numpy.ndarray[numpy.float64\PYZus{}t, ndim=1] b):
              cdef int i, n = len(a)
              b[0] = a[0]
              for i from 1 \PYZlt{}= i \PYZlt{} n:
                  b[i] = b[i\PYZhy{}1] + a[i]
              return b
\end{Verbatim}

    \begin{Verbatim}[commandchars=\\\{\}]
{\color{incolor}In [{\color{incolor}206}]:} \PY{n}{timeit} \PY{n}{cy\PYZus{}dcumsum2}\PY{p}{(}\PY{n}{a}\PY{p}{,}\PY{n}{b}\PY{p}{)}
\end{Verbatim}

    \begin{Verbatim}[commandchars=\\\{\}]
1000 loops, best of 3: 552 µs per loop
    \end{Verbatim}

    \subsubsection{Further reading}\label{further-reading}

    \begin{itemize}
\itemsep1pt\parskip0pt\parsep0pt
\item
  http://cython.org
\item
  http://docs.cython.org/src/userguide/tutorial.html
\item
  http://wiki.cython.org/tutorials/numpy
\end{itemize}


    % Add a bibliography block to the postdoc
    
    
    
    \end{document}
